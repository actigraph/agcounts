%% BioMed_Central_Tex_Template_v1.06
%%                                      %
%  bmc_article.tex            ver: 1.06 %
%                                       %

%%IMPORTANT: do not delete the first line of this template
%%It must be present to enable the BMC Submission system to
%%recognise this template!!

%%%%%%%%%%%%%%%%%%%%%%%%%%%%%%%%%%%%%%%%%%%%%%%%%%%%%%%%%%%%%%%%%%%%%
%%                                                                 %%
%% For instructions on how to fill out this Tex template           %%
%% document please refer to Readme.html and the instructions for   %%
%% authors page on the biomed central website                      %%
%% http://www.biomedcentral.com/info/authors/                      %%
%%                                                                 %%
%% Please do not use \input{...} to include other tex files.       %%
%% Submit your LaTeX manuscript as one .tex document.              %%
%%                                                                 %%
%% All additional figures and files should be attached             %%
%% separately and not embedded in the \TeX\ document itself.       %%
%%                                                                 %%
%% BioMed Central currently use the MikTex distribution of         %%
%% TeX for Windows) of TeX and LaTeX.  This is available from      %%
%% http://www.miktex.org                                           %%
%%                                                                 %%
%%%%%%%%%%%%%%%%%%%%%%%%%%%%%%%%%%%%%%%%%%%%%%%%%%%%%%%%%%%%%%%%%%%%%

%%% additional documentclass options:
%  [doublespacing]
%  [linenumbers]   - put the line numbers on margins

%%% loading packages, author definitions

%\documentclass[twocolumn]{bmcart}% uncomment this for twocolumn layout and comment line below
\documentclass{bmcart}

%%% Load packages
\usepackage{url}
\usepackage{amsthm,amsmath}
\RequirePackage{natbib}
%\RequirePackage[authoryear]{natbib}% uncomment this for author-year bibliography
%\RequirePackage{hyperref}
\usepackage[utf8]{inputenc} %unicode support
%\usepackage[applemac]{inputenc} %applemac support if unicode package fails
%\usepackage[latin1]{inputenc} %UNIX support if unicode package fails

%%%%%%%%%%%%%%%%%%%%%%%%%%%%%%%%%%%%%%%%%%%%%%%%%
%%                                             %%
%%  If you wish to display your graphics for   %%
%%  your own use using includegraphic or       %%
%%  includegraphics, then comment out the      %%
%%  following two lines of code.               %%
%%  NB: These line *must* be included when     %%
%%  submitting to BMC.                         %%
%%  All figure files must be submitted as      %%
%%  separate graphics through the BMC          %%
%%  submission process, not included in the    %%
%%  submitted article.                         %%
%%                                             %%
%%%%%%%%%%%%%%%%%%%%%%%%%%%%%%%%%%%%%%%%%%%%%%%%%

\def\includegraphic{}
\def\includegraphics{}

%%% Put your definitions there:
\startlocaldefs
\endlocaldefs

%%% Begin ...
\begin{document}

%%% Start of article front matter
\begin{frontmatter}

	\begin{fmbox}
		\dochead{Software}

		%%%%%%%%%%%%%%%%%%%%%%%%%%%%%%%%%%%%%%%%%%%%%%
		%%                                          %%
		%% Enter the title of your article here     %%
		%%                                          %%
		%%%%%%%%%%%%%%%%%%%%%%%%%%%%%%%%%%%%%%%%%%%%%%

		\title{Quantification of Acceleration as Activity Counts in
			ActiGraph Wearables}

		%%%%%%%%%%%%%%%%%%%%%%%%%%%%%%%%%%%%%%%%%%%%%%
		%%                                          %%
		%% Enter the authors here                   %%
		%%                                          %%
		%% Specify information, if available,       %%
		%% in the form:                             %%
		%%   <key>={<id1>,<id2>}                    %%
		%%   <key>=                                 %%
		%% Comment or delete the keys which are     %%
		%% not used. Repeat \author command as much %%
		%% as required.                             %%
		%%                                          %%
		%%%%%%%%%%%%%%%%%%%%%%%%%%%%%%%%%%%%%%%%%%%%%%

		\author[
			addressref={actigraph},
			% id's of addresses, e.g. {aff1,aff2}
			corref={actigraph},
			% id of corresponding address, if any
			% noteref={n1},                % id's of article notes, if any
			email={ali.neishabouri@theactigraph.com}
			% email address
		]{\inits{AN}\fnm{Ali} \snm{Neishabouri}}
		\author[
			addressref={actigraph},
			email={joe.nguyen@theactigraph.com}
			% email address
		]{\inits{JN}\fnm{Joe} \snm{Nguyen}}
		\author[
			addressref={actigraph, mgh},
			email={johnsam@alum.mit.edu}
			% email address
		]{\inits{JS}\fnm{John} \snm{Samuelsson}}
		\author[
			addressref={actigraph},
			email={tyler.guthrie@theactigraph.com}
			% email address
		]{\inits{JS}\fnm{Tyler} \snm{Guthrie}}
		\author[
			addressref={actigraph},
			email={matt.biggs@theactigraph.com}
			% email address
		]{\inits{MB}\fnm{Matt} \snm{Biggs}}
		\author[
			addressref={actigraph},
			email={jeremy.wyatt@theactigraph.com}
			% email address
		]{\inits{JW}\fnm{Jeremy} \snm{Wyatt}}
		\author[
			addressref={actigraph},
			email={doug.cross@theactigraph.com}
			% email address
		]{\inits{DC}\fnm{Doug} \snm{Cross}}
		\author[
			addressref={hsph},
			email={mkaras@hsph.harvard.edu}
			% email address
		]{\inits{MK}\fnm{Marta} \snm{Karas}}
		\author[
			addressref={kolinska,granada},
			email={jairo.hidalgo.migueles@ki.se}
			% email address
		]{\inits{JHM}\fnm{Jairo H.} \snm{Migueles}}
		\author[
			addressref={actigraph,mgh},
			email={sheraz.khan@theactigraph.com}
			% email address
		]{\inits{SK}\fnm{Sheraz} \snm{Khan}}
		\author[
			addressref={actigraph},
			email={christine.guo@theactigraph.com}
			% email address
		]{\inits{CCG}\fnm{Christine C.} \snm{Guo}}

		%%%%%%%%%%%%%%%%%%%%%%%%%%%%%%%%%%%%%%%%%%%%%%
		%%                                          %%
		%% Enter the authors' addresses here        %%
		%%                                          %%
		%% Repeat \address commands as much as      %%
		%% required.                                %%
		%%                                          %%
		%%%%%%%%%%%%%%%%%%%%%%%%%%%%%%%%%%%%%%%%%%%%%%

		\address[id=actigraph]{%                           % unique id
			\orgname{ActiGraph LLC},
			% university, etc
			\street{49 East Chase St.},
			%
			\postcode{FL 32502}
			% post or zip code
			\city{Pensacola},
			% city
			\cny{US}
			% country
		}
		\address[id=mit]{%
			\orgname{Massachusetts Institute of Technology},
			\street{77 Massachusetts Avenue},
			\postcode{MA 02139}
			\city{Cambridge},
			\cny{US}
		}
		\address[id=hsph]{%
			\orgname{Department of Biostatistics, Harvard T.H. Chan
				School of Public Health, Harvard University},
			\street{677 Huntington Avenue},
			\postcode{MA 02115}
			\city{Boston},
			\cny{US}
		}
		\address[id=kolinska]{%
			\orgname{Department of Biosciences and Nutrition,
				Karolinska Institutet},
			\city{Huddinge},
			\cny{Sweden}
		}
		\address[id=granada]{%
			\orgname{PROFITH “PROmoting FITness and Health through
				physical activity”
				Research Group, Sport and Health University
				Research Institute (iMUDS),
				Department of Physical Education and Sports,
				Faculty of Sport Sciences,
				University of Granada},
			\city{Granada},
			\cny{Spain}
		}
		\address[id=mgh]{%
			\orgname{Athinoula A. Martinos Center for Biomedical
				Imaging, MGH/MIT/Harvard},
			\street{149, 13th ST}
			\postcode{MA 02129}
			\city{Boston},
			\cny{US}
		}

		%%%%%%%%%%%%%%%%%%%%%%%%%%%%%%%%%%%%%%%%%%%%%%
		%%                                          %%
		%% Enter short notes here                   %%
		%%                                          %%
		%% Short notes will be after addresses      %%
		%% on first page.                           %%
		%%                                          %%
		%%%%%%%%%%%%%%%%%%%%%%%%%%%%%%%%%%%%%%%%%%%%%%

		\begin{artnotes}
			%\note{Sample of title note}     % note to the article
			%\note[id=n1]{Equal contributor} % note, connected to author
		\end{artnotes}

	\end{fmbox}% comment this for two column layout

	\begin{abstractbox}

		\begin{abstract} % abstract
			\parttitle{Background}
			%if any
			Digital clinical measures based on data collected by
			wearable devices have seen rapid growth in both clinical trials and healthcare.
			The widely-used measures based on wearables are epoch-based physical activity
			counts using accelerometer data. Even though activity counts have been the
			backbone of thousands of clinical and epidemiological studies, there are large
			variations of the algorithms that compute counts and their associated
			parameters – many of which have often been kept proprietary by device
			providers. This lack of transparency has hindered comparability between studies
			using different devices and limited their broader clinical applicability.
			ActiGraph devices have been the most-used wearable accelerometer devices for
			over two decades.

			\parttitle{Results}
			%if any
			Recognizing the importance of data transparency,
			interpretability and interoperability to both research and clinical use, we
			here describe the detailed counts algorithms of five generations of ActiGraph
			devices going back to the first AM7164 model, and publish the current counts
			algorithm in ActiGraph’s ActiLife and CentrePoint software as a standalone
			Python package for research use.

			\parttitle{Conclusion}
			We believe that this material will provide a useful
			resource for the research community, accelerate digital health science and
			facilitate clinical applications of wearable accelerometry.
		\end{abstract}

		%%%%%%%%%%%%%%%%%%%%%%%%%%%%%%%%%%%%%%%%%%%%%%
		%%                                          %%
		%% The keywords begin here                  %%
		%%                                          %%
		%% Put each keyword in separate \kwd{}.     %%
		%%                                          %%
		%%%%%%%%%%%%%%%%%%%%%%%%%%%%%%%%%%%%%%%%%%%%%%

		\begin{keyword}
			\kwd{accelerometer}
			\kwd{count}
			\kwd{actigraphy}
			\kwd{IMU}
		\end{keyword}

		% MSC classifications codes, if any
		%\begin{keyword}[class=AMS]
		%\kwd[Primary ]{}
		%\kwd{}
		%\kwd[; secondary ]{}
		%\end{keyword}

	\end{abstractbox}
	%
	%\end{fmbox}% uncomment this for twcolumn layout

\end{frontmatter}

%%%%%%%%%%%%%%%%%%%%%%%%%%%%%%%%%%%%%%%%%%%%%%
%%                                          %%
%% The Main Body begins here                %%
%%                                          %%
%% Please refer to the instructions for     %%
%% authors on:                              %%
%% http://www.biomedcentral.com/info/authors%%
%% and include the section headings         %%
%% accordingly for your article type.       %%
%%                                          %%
%% See the Results and Discussion section   %%
%% for details on how to create sub-sections%%
%%                                          %%
%% use \cite{...} to cite references        %%
%%  \cite{koon} and                         %%
%%  \cite{oreg,khar,zvai,xjon,schn,pond}    %%
%%  \nocite{smith,marg,hunn,advi,koha,mouse}%%
%%                                          %%
%%%%%%%%%%%%%%%%%%%%%%%%%%%%%%%%%%%%%%%%%%%%%%

%%%%%%%%%%%%%%%%%%%%%%%%% start of article main body
% <put your article body there>

%%%%%%%%%%%%%%%%
%% Background %%
%%
\section*{Background}
The rapid advances in computing and micro-electromechanical systems (MEMS)
technology have created new opportunities for monitoring health in people's
free-living environment over extended time periods. In particular, as wearable
devices are capable of quantifying patients’ behaviors continuously with
minimal burden, they can provide more comprehensive, ecological and objective
health information than what is possible with conventional in-clinic and
questionnaire-based assessments \cite{chinapaw_physical_2010,
	lee_validity_2011, van_poppel_physical_2010}. For these reasons, the use of
wearable devices for clinical purposes has seen rapid expansion in recent years
and is expected to grow further in the upcoming years
\cite{strain_wearable-device-measured_2020}. One of the most used outcomes
provided by wearable devices is accelerometry-based physical activity measures.
Physical activity measures have been used in clinical trials and shown
superiority in detecting treatment related changes over conventional
endpoints \cite{nathan_randomized_2020}. Progress has also been made in the
qualification of digital endpoints based on counts of physical activity with
both US and EU in respiratory and cardiovascular diseases
\cite{european_medicines_agency_qualification_2018,
	european_medicines_agency_qualification_2019}.

While the origin of the term “count” is not documented, it likely goes back to
the modification and use of watches for measuring activity, specifically
hyperactivity in children where pedometers were too large
\cite{bell_adaptation_1968, johnson_hyperactivity_1971,maccoby_activity_1965,
	schulman_objective_1959}. This was achieved by a winding weight that was free
to pivot about its axis and therefore sensitive to acceleration while being
connected to the hands of the watch via gears, causing the minutes and hours
hands to tick when the watch experienced acceleration and thus allowing for
reading the accumulated “counts” of movements from the watch the same way as
one would read the time \cite{schulman_objective_1959}. As electronics
technology evolved, solid-state analog-to-digital converters became widely
available and allowed for enhanced acceleration measurements and storing the
activity data in a digital format inside the watch
\cite{redmond_observations_1985, taylor_new_1982, tryon_fully_1996}. The first
activity watch provided by ActiGraph was model AM7164 (ActiGraph, Pensacola
(FL), USA) that used a uniaxial piezoelectrical accelerometer in the form of a
mechanical lever \cite{tryon_fully_1996}. To remove artifacts unrelated to
human movement, subsecond-level measurements of acceleration (expressed in \textit{g})
were passed through an analog band-pass filter before it was sampled at 10 Hz
and quantized by an 8-bit analog-to-digital converter, thus yielding 256
distinct levels of acceleration, each of these levels considered to be 1 count.
These were then summed within given time intervals, or “epochs”, and the
resulting counts were stored on the device (hence, each epoch could contribute
up to 255 counts to the total).

The counts unit is thus a measure that quantifies acceleration within a time
interval, or “epoch”, with one epoch typically being 10-60 seconds long. The
reliance on epoch-based counts was a necessity in earlier models, due to the
limitation of on-board storage and battery capacity. While this limitation was
overcome with hardware improvements and ActiGraph has been able to provide
multi-day raw data since 2010, digital clinical measures and validation
evidence continue to evolve around counts.  Because the algorithms that
transform the raw accelerometer data into counts vary across devices, many of
which have also been held proprietary, digital measures based on counts are
device dependent, making it difficult to compare results across
clinical studies and establish reproducibility and validation evidence
\cite{jul_4_comparing_2001,paul_comparison_2007, straker_translation_2012}. The
obscurity of the count algorithms has also led to the common misconception that
“counts” is a universal unit of measurement that is the same across devices,
where even with an algorithm such as the one described below, variations
in parameters such as the length of each epoch or the precision of the
quantization process can lead to vastly different counts. Therefore, the
advances of digital health science could be greatly facilitated with a higher
level of algorithm transparency.

ActiGraph devices (ActiGraph, Pensacola, FL, USA) have been the most-used
wearable accelerometry, or actigraphy, devices for over two decades, with more
than 20,000 papers published using ActiGraph devices by the end of 2021. While
the original methods for computing counts in the earlier models have been
published \cite{tryon_fully_1996}, the detailed algorithms and their evolution
over the years have not been made public. Several studies have attempted to
reverse engineer the ActiGraph counts algorithms or relate them to the counts
algorithms of other devices \cite{brond_generating_2017, brond_sampling_2016,
	paul_comparison_2007,straker_translation_2012}. While these studies have been
critically important to gain further understanding of acceleration counts and
how it relates across devices, they have been limited by the lack of access to
the ActiGraph firmware and software code and device documentation. In this
article, we address this gap by presenting an overview of the counts algorithms
of ActiGraph devices along with a detailed description of the counts algorithm
in both the ActiLife software (ActiGraph, Pensacola, FL, USA) and the
CentrePoint cloud service (ActiGraph, Pensacola, FL, USA). An open-source
Python package is also made available for use by the research community.  By
doing so, we hope to facilitate reproducibility efforts and enhance
transparency in the field of wearable accelerometry and accelerate its clinical
use.

\section*{Implementation}
Internal documentation of ActiGraph devices was reviewed to catalog the
different processing steps to convert from raw data to counts. The code for
converting raw data into counts in the ActiLife software was inspected along
with the firmware documentation for ActiGraph models wGT3X-BT, GT9X, and
CentrePoint Insight Watch (CPIW). The ActiLife counts algorithm was translated
into a standalone Python package which converts raw data into counts at
user-defined sampling frequencies and epoch lengths. The Python code was then
converted into pseudocode, i.e. human-understandable basic instructions that
can easily be translated back into any programming language. The code and
pseudocode thus apply to wGT3X-BT, GT9X, and CPIW, although CPIW requires a
down-sampling step before the computational pipeline described in the
pseudocode can be applied, since the sampling rate is different in CPIW
compared to the wGT3X-BT and GT9X models.

The Python code (Python 3.7.6) was validated to the counts algorithm in
ActiLife (version 6.13.4) and in CentrePoint (version 3.29.0) by comparing
counts output from the Python code with that of the ActiLife software and
CentrePoint using the same raw input data, sampling frequency and epoch length.
The raw input data were simulated Gaussian white noise and therefore consisted
of a broad range of frequencies. The code was tested for all admissible
sampling frequencies of these models (30, 40, \ldots, 100 Hz) and varying epoch
durations.

\section*{Results}
We first present here a flowchart overview of the processing pipelines to
derive counts across five generations of ActiGraph models
(Fig.~\ref{fig:pipeline}). It is important to note that all counts algorithms
outlined here produce counts after band pass filtering the raw signal. The
conversion of the raw data into counts depends therefore not only on the
amplitude of the raw data but also on its spectral content. In the CPIW model
(bottom row in Figure~\ref{fig:pipeline}), the accelerometer applies an
anti-aliasing low-pass filter before the signal is sampled to respect Nyquist's
theorem \cite{shannon_communication_1949}, and both the low-pass -3 dB cutoff
frequency and the sampling frequency are adjustable; the possible output data
rates are 32, 64, 128, and 256 Hz and the associated low-pass cutoff
frequencies are 16, 16, 32, and 64 Hz, respectively. The process of converting
raw data to counts (highlighted in bold in Fig.~\ref{fig:pipeline}) for the
newer models wGT3X-BT, GT9X, and CPIW are further elucidated below
(Fig.~\ref{fig:alg}).

\begin{enumerate}

	\item{Only for CPIW; resample data to 30 Hz (from 32, 64, 128 or 256
	            Hz). Otherwise assert that sampling frequency is between 30 and 100 Hz and a
	            multiple of 10 Hz.}
	\item{If sampling frequency is not a multiple of 30, then up-sample by
	            a factor of 3 to make it a multiple of 30 and low pass filter. This is achieved
	            by the following steps:
	            \begin{align*}
		            v       & = \textrm{array of size } (3 \times
		            \textrm{length of raw signal}) \textrm{ filled with } 0 \\
		            w       & = v
		            \\
		            v_{3\times i} & = \textrm{raw signal}_i ~\forall i\in
		            \left[ 0, \ldots, \textrm{size}(\textrm{raw signal})\right]            \\
		            w_i     & = \frac{3\pi}{\pi+3\times2}
		            v_i+\frac{3\pi}{\pi+3\times2}v_{i-1} +\frac{2\times3-\pi}{\pi+2\times3}w_{i-1} \forall i
	            \end{align*}
	      }
	\item{Down-sample to 30 Hz. Since the signal is a multiple of 30 Hz,
	            the down-sampling is straightforward by keeping every $m^\textrm{th}$ sample,
	            where m is the up-sampled frequency divided by 30;
	            \begin{align*}
		            m   & = \frac{\textrm{up-sampled frequency}}{30} \\
		            x_i & = w_{m\times i}~\forall i
	            \end{align*}
	      }
	\item{
	Band-pass filter the down-sampled signal. This is done by firstly
	defining the filter (a 7th order IIR filter), which can be described by the
	rational transfer function in the z-transform domain as:

	$$\hat{X}(z)=\frac{\sum_{i=0}^{7}b_iz^{-i}}{\sum_{i=0}^{7}a_iz^{-i}}X(z)$$
	where $a_0=1$,	$a_1=-3.63$, $a_2=5.04$, $a_3=-3.10$, $a_4=0.506$,
	$a_5=0.324$, $a_6=-0.157$, $a_7=0.0195$, and $b_0=-0.00934$, $b_1=-0.0255$,
	$b_2=-0.00424$, $b_3=0.0442$, $b_4=0.0365$, $b_5=-0.0119$, $b_6=-0.0229$,
	$b_7=-0.00679$ \cite{tryon_fully_1996}. These coefficients have been rounded to
	3 significands – please refer to the public code for the exact numbers. The
	initial condition that generates a steady state to a step response of this
	filter is then found, and the down-sampled signal $x_i$ is filtered using this
	initial state and the filter described above, resulting in the filtered signal
	$\hat{x}_i$.}
	\item{Rescale the filtered signal by a factor $a$ ($a=\frac{3.0 /
			    4096.0}{2.6 / 256.0} \times 237.5 \approx 17.127404$), in order
	            to replicate the range of the AM7164.
	            $$\tilde{x}_i=a\hat{x}_i$$}
	\item{Rectify the rescaled signal: $$\bar{x}_i =|\tilde{x}_i|$$}
	\item{Threshold the rectified signal so that all entries greater than
	            128 is set to 128 and all entries smaller than 4 are set to 0:
	            \[
		            \overrightarrow{x}_i = \begin{cases}
			            128                       & (\bar{x}_i>128)
			            \\
			            \textrm{floor}(\bar{x}_i) &
			            (4\leq\bar{x}_i\leq128)                           \\
			            0                         & (\bar{x}_i<4)
		            \end{cases}
	            \]
	      }
	\item{Further down-sample the signal and low-pass filter to 10 Hz by a
	non-overlapping moving average:

	$$\check{x}_j=\left\{\textrm{floor}(\frac{\overrightarrow{x}_i+\overrightarrow{
				x}_{i+1}+\overrightarrow{x}_{i+2}}{3})\right\}_{i=0,3,6,\ldots}$$
	}
	\item{Finally, find the counts by summing the down-sampled signal
	            within the predefined epoch length l (seconds) for each axis:

	            $$\textrm{counts}_i=\sum_{j=i\cdot10\cdot l}^{(i+1)\cdot10\cdot l}\check{x}_j$$}

\end{enumerate}

The python code was validated to the ActiLife software and CentrePoint service
using synthetic raw data as described in the methods section. Validation was
performed on four cases with 1000 simulations each: corresponding to
combinations of raw data at 40 or 30 Hz sampling rate and epoch of 10 or 30 sec
duration. We did not find any differences between the values returned by any of
the three implementations.

\section*{Discussion}
Remote clinical assessments based on wearable devices have the potential to
transform clinical development and care
\cite{strain_wearable-device-measured_2020}. This potential requires robust
validation evidence and can be facilitated with a greater level of algorithm
transparency. Here we provided an overview of the counts algorithms of five
generations of ActiGraph devices (Fig.~\ref{fig:pipeline}), released a Python
package that replicates the counts algorithm in ActiGraph’s ActiLife software
under an open source license, validated the Python code to the output from
ActiLife (Fig.~\ref{fig:results}) and detailed the numerical steps of arriving
at counts in pseudocode (Fig.~\ref{fig:alg}). We believe that this material
will be an important and useful resource for the research community.

Several kinds of counts algorithms have been used in the literature: Time above
threshold, zero-crossing and digital integration \cite{jul_4_comparing_2001}.
While the time above threshold and zero-crossing techniques have mostly been
used in older devices, all ActiGraph models have used the digital integration
technique, which is thought to be superior to the other counts techniques
\cite{redmond_observations_1985} in terms of component size, power consumption,
and accuracy in presence of noise. The processing steps of this technique
differed between devices, as successive generations of ActiGraph devices made
use of the latest advances in hardware and software technology. The computation
processes have gradually evolved to being done
in the CentrePoint cloud, and less so in the device microprocessor. Care has been
taken in order to maintain comparability between counts extracted from
different devices, so that the activity counts computed from the same
acceleration input would stay consistent across generations of hardware and
software platforms \cite{cain_comparison_2013, feito_evaluation_2012,
	grydeland_comparison_2014} although, due to the vastly different technologies
involved, differences have been found notably in minutes of vigorous activity
per day \cite{whitaker_comparison_2018}. This attention to data compatibility
due to platform upgrades in clinical investigation has also been noted in the
recent FDA draft guidelines on digital health technologies
\cite{research_digital_2021}.

\citet{brond_generating_2017} tried to reverse engineer the ActiLife counts
algorithm by reproducing the counts output from raw data input. The ActiGraph
digital band-pass filter was approximated using Matlab’s \emph{invfreqz}
function. The reverse engineered counts algorithm was then applied to raw data
from a different activity monitor; Axivity AX3 (Axivity, Newcastle UK) and they
found generally high concordance between this output and the output from
ActiLife applied to raw data from the GT3X+ monitor (not statistically
significant from zero for mechanical validation experiment and 2.2\% mean
difference from free-living conditions). Figure 1 in
\citet{brond_generating_2017} outlines the steps in the derived counts
algorithm which corresponds rather well with Figure~\ref{fig:alg} in the
present paper which details the ActiLife counts algorithm, although many of the
computational steps happen in a different order. Furthermore, the ActiLife
digital band-pass filter that was approximated in \citet{brond_generating_2017}
is different from the one implemented in ActiLife; the ActiLife band-pass
filter is a $7^\textrm{th}$ order IIR filter while the one estimated in their
study is a 20-order filter.

With the many versions of activity counts and additional open source activity
metrics, such as Euclidean Norm Minus One (ENMO), Activity Index (AI)
\cite{bai_activity_2016, hees_separating_2013}, it is important to apply the
cut points developed for the right type of activity counts.  On the other hand,
since there is generally high correlation between these activity metrics, it is
feasible to compare the findings using different metrics with relatively simple
conversion factors.  In an experiment under free-living condition,
\citet{paul_comparison_2007} showed that there was a significant difference in
the counts output between the ActiGraph AM7164 monitor and the ones of Actical
(MM; Mini-Mitter Co.) and it can be reduced significantly by the conversion
equations they developed. \citet{straker_translation_2012} also presented a
linear conversion equation relating the vertical component of the ActiGraph
GT3X model with the Actical counts.

To further facilitate the use and comparability of ActiGraph counts, we here
also publish the codes in an open source Python package. The code validation
showed perfect correspondence between ActiLife, CenterPoint, and the published
Python code (see Fig.~\ref{fig:results}). Although we expect and indeed observe
small differences between the intermediary results of different implementations
due to the nature of working with floating point numbers, these differences had no
effect on the final results of these algorithms.

\section*{Conclusion}
Quantifying acceleration in ‘counts’ started due to technical limitations in
the earliest wearable devices but has demonstrated its value for clinical
research and is still omnipresent in the field of wearable accelerometry.
Counts is not a universal unit, and the relationship between raw acceleration
data and counts is complicated and varies between device models. Therefore,
detailed understanding of the computation from raw accelerometer data to counts
is necessary for interpreting clinical outcomes across studies and advancing
the clinical application of wearable technology. In addition, proper
demonstration and documentation of signal processing steps are necessary in
obtaining regulatory agreement in the use of digital health technologies tools
in clinical trials and/or medical devices. We hope that by publishing and
detailing the counts algorithms of ActiGraph devices in this study, we could
help accelerate digital health science and facilitate clinical applications of
wearable accelerometry.

\section*{List of Abbreviations}
\begin{itemize}
\item{MEMS: Micro-ElectroMechanical  Systems}
\item{CPIW: CenterPoint Insight Watch}
\item{FDA: U.S. Food and Drug Administration}
\item{ENMO: Euclidean norm minus one}
\end{itemize}

%%%%%%%%%%%%%%%%%%%%%%%%%%%%%%%%%%%%%%%%%%%%%%
%%                                          %%
%% Backmatter begins here                   %%
%%                                          %%
%%%%%%%%%%%%%%%%%%%%%%%%%%%%%%%%%%%%%%%%%%%%%%
\begin{backmatter}
	\section*{Declarations}
	\subsection*{Ethics approval and consent to participate}
	Not applicable
	\subsection*{Consent for publication}
	Not applicable
	\subsection*{Availability of data and materials}
	The count algorithm and the code to reproduce and plot the validation
	data presented in Figure~\ref{fig:results} are publicly available at
	\url{https://github.com/actigraph/agcounts}. The code is also available as a
	python package. The parser for GT3x files is available at
	\url{https://github.com/actigraph/pygt3x}.

	All code is released as-is, with no guarantee of functionality.
	\subsection*{Competing interests}
	Ali Neishabouri, Joe Nguyen, Tyler Guthrie, Matt Biggs, Jeremy Wyatt,
	Doug Cross, and Christine Guo are employed by ActiGraph. Sheraz Khan and John
	Samuelsson are providing consulting services to ActiGraph.
	\subsection*{Funding}
	Jairo H. Migueles is supported by the Swedish Research Council for
	Health, Working Life and Welfare (FORTE) (dnr: 2021-00036). Marta Karas is supported by the National Institute of Mental Health (NIMH) under award number 5U01MH116928.

	\subsection*{Authors' contributions}
	AN implemented the final version of the algorithm, and took the lead for finalising the manuscript. JS wrote the first version of this manuscript and the algorithm. JN, TG, MB, JW and DC provided information on existing implementations and the history of the algorithm. JN also provided ground truth values used for testing the new implementation. MK, JHM, SK and CCG provided scientific input and helped on the interpretation of results.  All authors provided critical feedback and helped shape the research, analysis and manuscript.
	\subsection*{Acknowledgements}
	Not Applicable

	%%%%%%%%%%%%%%%%%%%%%%%%%%%%%%%%%%%%%%%%%%%%%%%%%%%%%%%%%%%%%
	%%                  The Bibliography                       %%
	%%                                                         %%
	%%  Bmc_mathpys.bst  will be used to                       %%
	%%  create a .BBL file for submission.                     %%
	%%  After submission of the .TEX file,                     %%
	%%  you will be prompted to submit your .BBL file.         %%
	%%                                                         %%
	%%                                                         %%
	%%  Note that the displayed Bibliography will not          %%
	%%  necessarily be rendered by Latex exactly as specified  %%
	%%  in the online Instructions for Authors.                %%
	%%                                                         %%
	%%%%%%%%%%%%%%%%%%%%%%%%%%%%%%%%%%%%%%%%%%%%%%%%%%%%%%%%%%%%%

	% if your bibliography is in bibtex format, use those commands:
	\bibliographystyle{plainnat} % Style BST file (bmc-mathphys, vancouver, spbasic).
	\bibliography{bmc_article}	% Bibliography file (usually '*.bib' )
	% for author-year bibliography (bmc-mathphys or spbasic)
	% a) write to bib file (bmc-mathphys only)
	% @settings{label, options="nameyear"}
	% b) uncomment next line
	%\nocite{label}

	% or include bibliography directly:
	% \begin{thebibliography}
	% \bibitem{b1}
	% \end{thebibliography}

	%%%%%%%%%%%%%%%%%%%%%%%%%%%%%%%%%%%
	%%                               %%
	%% Figures                       %%
	%%                               %%
	%% NB: this is for captions and  %%
	%% Titles. All graphics must be  %%
	%% submitted separately and NOT  %%
	%% included in the Tex document  %%
	%%                               %%
	%%%%%%%%%%%%%%%%%%%%%%%%%%%%%%%%%%%

	%%
	%% Do not use \listoffigures as most will included as separate files

	\section*{Figures}
	\begin{figure}[h!]
		\caption{\csentence{Conceptual schematic of the computational
				pipeline for generating counts for five generations of ActiGraph accelerometer
				models (rows).}
			Raw data measures acceleration in free fall
			acceleration (g) units. Functions performed by the microprocessor, the
			accelerometer and in the CentrePoint cloud are highlighted in yellow, gray and
			orange, respectively. The analog band-pass filters are implemented using a
			series of cascaded op amp circuits, the -3dB cutoff frequency is shown in
			parenthesis. A/D is the analog-to-digital converter (white blocks), shown
			together with the bit size of the quantizer. The counts data blocks are
			highlighted in boldface.}
		\label{fig:pipeline}
	\end{figure}
	\begin{figure}[h!]
		\caption{\csentence{Flowchart showing the process of converting
				raw data into counts.}}
		\label{fig:alg}
	\end{figure}
	\begin{figure}[h!]
		\caption{\csentence{Comparison of counts generated by our
				python code and those generated by CenterPoint.} Each plot corresponds to a
			combination of data sampling rate and epoch length. From top to bottom: 30Hz
			and 30s, 40Hz and 30s, 30Hz and 10s, 40Hz and 10s.
			\label{fig:results}
		}
	\end{figure}

	%%%%%%%%%%%%%%%%%%%%%%%%%%%%%%%%%%%
	%%                               %%
	%% Tables                        %%
	%%                               %%
	%%%%%%%%%%%%%%%%%%%%%%%%%%%%%%%%%%%

\end{backmatter}
\end{document}